%%%% ABSTRACT
%%
%% Versão do resumo para idioma de divulgação internacional.

\begin{abstractutfpr}%% Ambiente abstractutfpr
The study of dynamic relationships between topological structures of a transport network and mobility patterns in this network is important for building smart solutions to problems of reliability, optimization, vulnerability and traffic forecast. However, one of the biggest challenge in transport planning and operation management is to deal with large volume of data often provided by geolocation sensors installed in vehicles. This work aims at modeling Curitiba transport system using a graph database built with information of daily transport operation available from an open data repository. The main contribution is the development of a computational framework for automatic building of the graph database from raw data of the repository. Illustrative results of several metrics for Curitiba transport are presented by showing the potential analysis provided by the proposed tool.

To define the keywords (and their corresponding portuguese in the \textit{resumo}) query in Authorities Catalog Topic term of the National Library, available at: \url{http://acervo.bn.br/sophia_web/index.html}.
\end{abstractutfpr}
