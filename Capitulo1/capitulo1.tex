%%%% CAPÍTULO 1 - INTRODUÇÃO
%%
%% Deve apresentar uma visão global da pesquisa, 
%% incluindo: breve histórico, importância e
%% justificativa da escolha do tema, delimitações
%% do assunto, formulação de hipóteses e objetivos
%% da pesquisa e estrutura do trabalho.

\definecolor{courb2020}{RGB}{0, 204,0}


%% Título e rótulo de capítulo (rótulos não devem conter caracteres especiais, acentuados ou cedilha)
\chapter{Introdução}\label{cap:introducao}

De acordo com ~\cite{Dem:20}, nos últimos anos, a população mundial tornou-se mais da metade urbana. É estimado que em 2025, as áreas urbanas ao redor do globo representem 58\% da população mundial, aumentando para mais dois terços em 2050~\cite{new:20}. 
Este expressivo aumento populacional e expansão urbana traz grandes desafios no planejamento urbano, nas mais diversas áreas, tais como saúde, infraestrutura e mobilidade. 
Com finalidades variadas, diariamente, milhares de pessoas ao redor do mundo utilizam transporte público para se locomover nas cidades, sendo que nas grandes metrópoles, as pessoas cobrem grandes distâncias e passam um tempo razoável em trânsito.
O relatório de 2020 sobre o cenário do Transporte Público no mundo divulgado pelo Moovit~\cite{Mov:20} (aplicativo usado diariamente por 865 milhões de passageiros de todo o mundo), mostra que no Brasil, aproximadamente 10\% dos usuários se deslocam por mais de 2 horas diárias em grandes cidades como Rio de Janeiro, Recife e São Paulo. Além disso, cerca de 40\% dos usuários esperam mais de 20 minutos por dia na estação de ônibus.
Citada como uma cidade referência na área de mobilidade urbana~\cite{CWBconhecida} e conhecida por investimentos na priorização do transporte público desde os anos 70, a cidade de Curitiba conta com uma Rede Integrada de Transporte (RIT), baseada em ônibus urbanos e que permite que o passageiro troque de linhas em determinados pontos com o custo de uma única tarifa. Atualmente, Curitiba é atendida por uma frota de 1410 ônibus operantes (mais reserva) e atende cerca de 1.389.731 passageiros por dia com 251 linhas de ônibus, 329 estações e 21 terminais~\cite{Cur:19}. 

A administração da rede pública de transporte de Curitiba é realizada pela Urbanização de Curitiba S/A (URBS), uma empresa de economia mista responsável pela fiscalização das empresas particulares prestadoras de serviços~\cite{URBS}. Através de levantamentos realizados junto à população, a qualidade do transporte, assim como rotas preferenciais, número de passageiros transportados e horários de maior demanda são estimados, bem como auxiliam no planejamento operacional do sistema (inclusão de novas linhas de ônibus e atualização da programação horária, por exemplo).

Diante dos fatores do crescimento populacional nas áreas urbanas e utilização massiva dos meios de transporte público, um dos maiores desafios da área de gerência, planejamento e operação do transporte é a coleta e manipulação das mais variadas fontes e volumes de dados disponíveis, geralmente provenientes de sensores de localização instalados nos veículos~\cite{wes:17}. 

Atualmente, o sistema de transporte coletivo de ônibus de Curitiba disponibiliza um repositório com dados abertos sobre sua operação\footnote{http://dadosabertos.c3sl.ufpr.br/}. Este repositório é frequentemente atualizado com dados brutos da operação provenientes de sensores e dispositivos móveis que capturam a movimentação dos ônibus ao longo do dia. Todavia, a utilização destes dados requer um modelo de dados adequado para análises mais complexas, que envolvem não apenas dados estáticos de configuração da rede de transporte, mas também dinâmicos de sua operação.
 

\textcolor{courb2020}{
Em geral, modelos estáticos de redes complexas são utilizados para caracterizar estatisticamente as redes de transporte a partir de métricas como centralidades de grau, proximidade e intermediação, distribuição dos comprimentos dos caminhos mínimos, coeficiente de agrupamento, dentre outras. Métricas de redes complexas foram utilizadas para caracterizar a conexão entre estações do sistema do sistema de transporte na Polônia~\cite{Sienkiewicz2005}, China~\cite{Xu2013} e Brasil~\cite{Izawa2017}. A caracterização temporal de redes de transporte de ônibus de Curitiba aparece em~\cite{curz:19} usando \emph{link streams} e de diversos modais de transporte na Grã-Bretanha em~\cite{Gallotti2015} usando redes \emph{multilayer}.
}

\ric{A introdução contextualiza o problema tratado: do que se trata, qual é o foco do trabalho e como a metodologia adotada se encaixa; feito isso, tem que dizer quais serão os pontos a serem atacados (modelagem dos dados de operação do transporte, segundo um grafo temporal; construção de uma base de dados para coletar, organizar e disponibilizar estes dados; análise dos dados de Curitiba e proposta de uma integração temporal das linhas de ônibus.}


\section{Motivação}

\textcolor{courb2020}{
Segundo~\cite{fer:04}, uma maior utilização do transporte público favorece a mitigação dos problemas de congestionamento, poluição, acidentes, desumanização e outros males que afligem as cidades modernas. 
Assim, estudar e propor melhorias e otimizações em mobilidade urbana torna-se importante à medida que as cidades enfrentam problemas no fornecimento de transporte público de qualidade.
}


\textcolor{courb2020}{
O estudo das relações dinâmicas entre as estruturas topológicas de uma rede de transporte e os padrões de mobilidade nesta rede se faz importante para a criação de soluções inovadoras para problemas de confiabilidade, otimização, vulnerabilidade e previsão de tráfego. 
}


\section{Objetivos}
 
\subsection{Geral}


\textcolor{courb2020}{
Este trabalho tem por objetivo modelar o sistema de transporte coletivo da cidade de Curitiba usando uma base de dados de grafos gerada a partir de um repositório de dados abertos com informações da operação diária do transporte. O modelo proposto é baseado em \cite{wach:19}, sendo o foco do trabalho a implementação deste modelo para Curitiba. A principal contribuição é o desenvolvimento de uma plataforma computacional para a construção automática da base de grafo a partir do repositório de dados abertos. A infraestrutura computacional e a transformação dos dados do repositório para a base de dados de grafo são detalhadas. 
}

 \subsection{Específico}
 
 \begin{itemize}
 \item Utilizar métricas de redes complexas para a avaliação da qualidade do sistema de transporte público.
 
 \item Aplicar uma metodologia para identificação de possíveis pontos para criação de terminais virtuais integrados.
 \end{itemize}
 
 
 
\section{Contribuição}
\label{sec:contrib}

A contribuição principal desta dissertação está na aplicação de um modelo temporal para a modelagem da operação de uma rede de transporte de ônibus. Em particular, da rede de transporte de Curitiba. Dado que estes modelos capturam informações temporais da operação, são utilizados para estabelecer uma metodologia de análise da implantação de uma integração temporal entre linhas de ônibus e seus impactos na operação. Não há muitos trabalhos na literatura que utilizem modelos formais da operação do transporte, cujos resultados possam ser usados para avaliar este tipo de integração temporal. Assim, os resultados desta dissertação são úteis como ponto de partida para ampliar o acesso ao serviço de transporte pela integração de linhas, assim como para definir políticas e serviços públicos de transporte de ônibus por gestores públicos. Além disso, tem-se também como contribuição o desenvolvimento de uma plataforma computacional...

Parte dos resultados desta dissertação foram publicados em~\cite{courb2020}:

\noindent
$\bullet$ PEIXOTO, Altieris; ROSA, Marcelo; LüDERS, Ricardo; FONSECA, Keiko. Plataforma computacional para construção de um banco de dados de grafo do sistema de transporte de Curitiba. In: Anais do IV Work. de Comp. Urbana. Porto Alegre, RS, Brasil: SBC, 2020. p. 125–137.
 
 

\section{Estrutura da dissertação}
\label{sec:estrutura}

O Capítulo~\ref{cap:revisaodaliteratura} apresenta os conceitos teóricos fundamentais utilizados na dissertação, além de uma discussão sobre trabalhos correlatos. No Capítulo~\ref{cap:materialemetodos}, é apresentada a metodologia de coleta, modelagem dos dados e a proposta de análise da integração temporal, assim como a plataforma computacional desenvolvida. Os resultados da operação atual da rede de transporte, assim como os resultados da análise de uma integração temporal de linhas de ônibus são apresentados no Capítulo~\ref{cap:resultadosediscussao}, seguido das conclusões da dissertação no Capítulo~\ref{cap:conclusoeseperspectivas}.

% O Capítulo 2 apresenta uma revisão da literatura sobre o transporte coletivo urbano, acessibilidade com foco no transporte público, indicadores de acessibilidade relacionados ao transporte. No capítulo 3, são descritos os procedimentos e ferramentas adotados para avaliar a acessibilidade do transporte para se chegar às US, assim como é apresentado um modelo para identificar os bairros prioritários em termos de investimento ou ajuste na rede de transporte. No capítulo 4, são apresentados os principais resultados encontrados para o estudo de caso realizado, dando destaque para as análises baseadas nas distâncias, tempo de viagem e sua relação com a renda média da população. No capítulo 5, são apresentadas as conclusões do trabalho e propostas de trabalho futuro.

% Na organização do artigo, a Seção~\ref{sec:fund} contém conceitos básicos e trabalhos relacionados. A Seção~\ref{sec:met} apresenta o modelo proposto. A Seção~\ref{sec:metr} define as métricas de redes complexas. A Seção~\ref{sec:impl} descreve a plataforma computacional desenvolvida. A Seção~\ref{sec:result} apresenta os resultados obtidos das métricas computadas para o transporte de Curitiba a partir do banco de dados de grafos. Conclusões e trabalhos futuros estão na Seção~\ref{sec:conclu}.