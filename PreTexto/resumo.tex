%%%% RESUMO
%%
%% Apresentação concisa dos pontos relevantes de um texto, fornecendo uma visão rápida e clara do conteúdo e das conclusões do trabalho.

\begin{resumoutfpr}%% Ambiente resumoutfpr
O estudo das relações dinâmicas entre as estruturas topológicas de uma rede de transporte e os padrões de mobilidade nesta rede se faz importante para a criação de soluções inovadoras para problemas de confiabilidade, otimização, vulnerabilidade e previsão de tráfego. Todavia, um dos maiores desafios da área de gerência, planejamento e operação do transporte é a manipulação de um grande volume de dados, geralmente provenientes de sensores de localização instalados nos veículos. Este trabalho tem por objetivo modelar o sistema de transporte coletivo da cidade de Curitiba usando uma base de dados de grafos gerada a partir de um repositório de dados abertos com informações da operação diária do transporte. A principal contribuição é o desenvolvimento de uma plataforma computacional para a geração automática da base de grafo a partir do repositório de dados abertos. Resultados ilustrativos de diversas métricas do transporte de Curitiba são apresentados, mostrando o potencial de análise da ferramenta.

Palavras-chave. Para definição das palavras-chave (e suas correspondentes em inglês no \textit{abstract}) consultar em Termo tópico do Catálogo de Autoridades da Biblioteca Nacional, disponível em: \url{http://acervo.bn.br/sophia_web/index.html}.
\end{resumoutfpr}
